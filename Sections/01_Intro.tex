\section*{1. Introduction}

With the advent of progressive miniaturization of electronic devices, it becomes necessary to describe the transport properties of small systems within a totally quantum mechanical framework. As a representative example, we just recall the double barrier resonant tunneling device (DBRTD) \cite{Esaki1974}, whose current-voltage ($I-V$) curve presents a well defined peak as shown in Fig.~\ref{fig:DBRTD} adapted from data in \cite{Goldman1987}. This behavior, in great departure from the usual monotonic features shown by macroscopic samples, can be attributed to quantum interference phenomena, which arises from the coherent dynamics of the carriers.

Let us characterize a ``device'', in a rough sense, as some specified region of material where the carriers spend a substantial part of the time. It might be characterized by a longitudinal length $L_{x}$ and a transverse cross section $L_{y} \times L_{z}= M\left(a^{2}\right)$, where $a$ is an atomic length scale. In order to manifest its wave nature, an electronic excitation must propagate quantum mechanically (i.e. phase coherently) between these boundaries. Typical interference phenomena occurs when any of these lengths is comparable with the deBroglie wavelength $\lambda_{\varepsilon}$. A condition for this quantum coherence is the weak coupling with the environmental degrees of freedom within the relevant time scales. This condition can often be achieved by shrinking the device length scales and/or lowering the temperatures until the minimum energy excitation $\Delta$ becomes large compared to the available thermal energy $k_{\mathrm{B}} T$. In the experiment of Fig.~\ref{fig:DBRTD} the applied voltage is the control parameter for the wave length of the electrons responsible for the transport. Therefore, the peaked $I-V$ curve of the DBRTD can be viewed much as resonant peaks in a Fabry-Perrot interferometer.

\begin{figure}[h]
  \includegraphics[width=0.95\columnwidth]{Figures/01_DRBTD.pdf}
  \caption{Scheme of a DBRTD and its experimental current-voltage characteristic. Results adapted from \cite{Goldman1987}. The points A, B and C in the curve correspond to the potential profile shown in the inset.}
  \label{fig:DBRTD}
\end{figure}

A great variety of solid-state devices developed in the last decade \cite{Altshuler1991} satisfy the general conditions for the quantum manifestations described above. More recently, the growing interest in understanding the conductance between two electrodes connected through a bridging molecule \cite{Reed2000} lies naturally in this category. In fact, discreteness of the molecular electronic energy levels is a clear manifestation of the quantum coherence of the electronic states and one is left with the question of how these states transport charge \cite{Mikkelsen1987}. The related issue of the propagation of a charge density excitation has been tackled by physical-chemists who, for a long time, have been dealing with problems as diverse as conducting polymers \cite{Heeger1988}, charge oscillation in the intermediate valence compounds \cite{Taube1981} [e.g. the Creutz-Taube ion $\left(\mathrm{NH}_{3}\right)_{5} \mathrm{Ru}$-pyz-Ru-$\left(\mathrm{NH}_{3}\right)_{5}{ }^{+5}$ where pyz stands for pyrazine] and electron transfer in photosynthetic systems \cite{Beratan1984}.

The traditional methods used to describe transport are not necessarily appropriate for mesoscopic and molecular systems. The semiclassical Boltzmann equation, the Kubo formalism and other traditional techniques were specially devised to describe bulk transport, where the thermodynamic limit is guaranteed and the linear response regime is the relevant one. Hence, in order to deal with transport in finite size ``samples'' with non-trivial geometrical constraints, and often in the presence of non-perturbative fields, a new perspective had to be adopted. It is clear that a general Schrödinger equation must provide the complete description. However, even in the simplest independent particle approximation, it seems difficult to include the complex boundary conditions imposed by the electrodes. A way out is to treat transport of carriers through the ``sample'' as a scattering problem from one electrode to the other. Therefore, the transmittance $T$ undertakes a most relevant role. This is essentially the conceptual framework on which Landauer's description of transport \cite{Landauer1957} is based. It involves a simple but conceptually new approach that inspired most of the advances in solid-state devices in recent years and is now clearing the way towards molecular electronics.

In this article, we attempt to present a self-contained operational and conceptual manual based on the lectures given by the authors at the 2nd Workshop on Mesoscopic Systems. We will attempt to summarize the methods used to evaluate and interpret the electronic structure of finite systems, as well as how they are extended to describe the coherent transport when they are connected to electrodes. While the effects of electron-electron interaction can become of utmost relevance in small systems \cite{Kastner1993,Devoret1992}, for didactic reasons we will not extend on this blooming field. Let us only mention that their understanding \cite{Kouwenhoven1997} is built upon the methods and ideas we describe in this review.

The material is presented as follows: Section~2 is devoted to Landauer's ideas, and represents the backbone over which the formalism is built. Section~3 introduces the decimation method to calculate transmittances and electronic structure of devices and molecules. Section~4 shows the connection between electronic-structure and transport obtained by Fisher and Lee. Their formula is derived and discussed there. Section~5 presents an overview of the known results of transport which can help the beginner to see the machinery in action. Section~6 is devoted to the introduction of decoherence into the formalism. Section~7 presents a brief outline of the time-dependent problem.

A number of the most simple calculations will be worked out with some detail, so the unfamiliar reader might acquire a first feel for the topic by noting the simplicity of the methodology. The most advanced results will be left merely indicated so that they can be reobtained by a motivated reader. Many of the examples presented here are completely original and have not been published before to our knowledge. Finally, let us stress that we do not attempt to review the literature exhaustively, but simply to present our personal pathway through the concepts involved. In particular, the issues of decoherence and time dependence are the object of current research and the reader might still create his own trek.
