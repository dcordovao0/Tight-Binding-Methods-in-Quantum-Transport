\section*{8. Final remarks}
We hope these lectures have been able to convey, at least partially, the essence of our message: In our attempt to understand the quantum world it is possible, and convenient, to make very simple models of nature. Those models, though not necessarily complete, can show us many effects that surprise our classical intuition and which would have been obscured\\
by other more ``complete'' descriptions. Every one of these surprises can give a new twist in the experimental research, which is indeed the ultimate truth, and eventually become the source of an innovative application.

In our toolbox to extract information from the proposed models the most important elements have been the decimation method and the Green's functions technique. Both are intimately connected with the renormalization group concept. As such they are fundamental in generating the models themselves, as they help us to identify the relevant variables. Simultaneously, our understanding of the quantum world has benefited from fresh approaches to long standing problems as that of the tunneling time. In the process of computing transport properties we have also learnt the deepest principles of statistical mechanics as to what is the origin of macroscopic irreversibility and established the conditions for the validity of the ergodic hypothesis. The fundamentals of decoherent processes in transport and their consequences are currently of intense interest. The lines of thought we have sketched here, have turned out to be very productive in this direction, and we hope they will continue being fruitful through the action of our students.
